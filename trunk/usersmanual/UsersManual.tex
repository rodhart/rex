\documentclass{article}
\usepackage[pdfborder={0 0 0}, colorlinks=true]{hyperref}
\author{CISC475/675 Spring 2010}

% \title{REX User's Manual}
\title{The Hitchhiker's Guide To REX}
\date{\today}

\begin{document}
\maketitle
\tableofcontents
\newpage

\section{Introduction}
The REX software package was designed as part of the CISC475 course,
to create a solution for Prof\. Harvey. Prof\. Harvey wishes to create
multiple versions of each exam for the classes he instructs. However,
this can be very time consuming, and as the number of exams created
increases, the amount of effort needed to be exerted to assure the
accuracy of the exams increases proportionally.

We sought to solve these problems for Prof\. Harvey, and created the
REX software as an educational exercise in real-world software
development practices.

The REX software runs without any installation requirements beyond the
need of an appropriate Java VirtualMachine. The REX software will take
UEF File and an ECF File as input, and non-deterministically save
some number of exam files and answer key files in a desired directory.

\section{Installation}
The REX software does not require any installation. It is suggested to
place the \textt{rex.jar} executable in a directory within the user
path on UNIX systems, but this is not necessary for proper operation
of the software.

\section{Usage}
After creating a UEF and ECF file \texttt{rex input.uef input.ecf}. 
REX will parse the contents of the UEF and ECF and generate the appropriate number of
output exams, as specified in the ECF file.

\section{The UEF File}

\section{The ECF File}

\section{Credits}

\end{document}
