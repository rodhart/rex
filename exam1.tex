\documentclass{exam}
 
\usepackage{fullpage}
\usepackage{amsmath}
\usepackage{comment}
\usepackage{graphicx}


\begin{document}


%print header
\examheader{CISC 106}{UEF Example}{Developed: 2010}

%example of a block

\begin{problem}{logical}{8.0}
In MATLAB the value 1 is true, and true is:
%:type logical
\begin{answers}
\answer[correct] any value that isn't zero %correct answer
\answer there are no boolean types in MATLAB
\answer 1
\answer[fixed] none of the above %fixed position
\answer 0
\end{answers}
\end{problem}
\begin{problem}{function call}{15.0}
Consider the function f:
  \begin{verbatim}
    function answer = f(a)
    answer = a+2;
    end
  \end{verbatim}
  What will be in x after the command:
  \begin{verbatim}>> x = f(4);
  \end{verbatim}
%:type function call
\begin{answers}
\answer[fixed] none of these %fixed position
\answer[fixed] error %fixed position
\answer[correct] 6 %correct answer
\answer f(4)
\answer 4
\end{answers}
\end{problem}
\begin{block}
\begin{block}{logical}
The following problems relate to logical situations.
\end{block}
\end{block}
\end{document}
